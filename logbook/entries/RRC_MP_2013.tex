\paragraph{Motivation}
The existing literature on the gains from inter-country interactions studies 
\textbf{trade in goods}, and \textbf{MP/FDI} in isolation.
The omission of combining these two interactions in analysis is important because 
trade agreements often combine tariff reductions and removal of barriers to MP.

\paragraph{Some Notations}
\begin{table}[h]
    \caption{Notations in \citep{Ramondo:2013}}
        \centering
        \begin{tabular}{c l} \toprule
            Notation & Meaning \\ \hline
            $i = \{1, 2, 3, \cdots, I\}$ & country of headquarter (parent) \\
            $l = \{1, 2, 3, \cdots, I\}$ & country of production \\
            $n = \{1, 2, 3, \cdots, I\}$ & country of destination \\
            $d_{nl} \geq 1$ & trade costs \\
            $h_{li}^g \geq 1$ & MP costs \\
            $v \in [0, 1]$ & tradable intermediate goods \\
            $u \in [0, 1]$ & non-tradable final goods \\ 
            \bottomrule
        \end{tabular}
        \begin{minipage}{0.6\textwidth}{\footnotesize
            \textsc{Notes}: $h_{li}^g \geq 1$ implies that home production is more efficient that those of foreign affiliates.}
        \end{minipage}
\end{table}


\subsubsection{Model}
\paragraph{MP Cost}
Unit cost of the MP input bundle for MP by $i$ in $l$:
\begin{equation}
 c_{li} = \bigg[ (1-a) (c_l h_{li})^{1 - \xi} + (a) (c_i d_{li})^{1 - \xi}
    \bigg]^{\frac{1}{1-\xi}}
\end{equation}

\paragraph{Productivity Distributions}
For a home country, she faces
a vector $\mathbf{z}_i^s = (z_{1i}, z_{2i}, \cdots z_{Ii}), s = g, f$ that is 
drawn independently across goods and countries from a Multivariate Frechet distribution:
\begin{equation}
    F_i\left(\mathbf{z}_i^s; T_i \right) 
    =\exp \left[-T_i\left(\sum_{l=1}^I\left(z_{l i}^s\right)^{\frac{-\theta}{1-\rho}}\right)^{1-\rho}\right]
\end{equation}
where $\rho$ governs the correlation of productivity across production locations.

\paragraph{Equilibrium Analysis}
Final goods are non-tradable, so $i$ must produce them in destination country $n$ to obtain positive market share.
Therefore, the price of final good $u$ in $n$ is 
\begin{equation*}
    p_n^f(u) = \min_i \frac{c_{ni}^f}{z_{ni}^f}
\end{equation*}

As $z_{ni}^f \sim F(z)$, 
the share of expenditure by country $n$ on final goods produced in country $n$ with country $i$ technologies is 
\begin{equation}
    \pi_{ni}^f = \frac{T_i (c_{ni}^f)^{-\theta}}{\sum_j T_j (c_{nj}^f)^{-\theta}}
\end{equation}


Compared to non-tradable final goods,
intermediate goods are tradable and can be imported from production countries that might differ 
from the home countries and destination countries $(i \neq l \neq n)$.

The price of intermediate good $v$ in $n$ is 
\begin{equation*}
    p_n^g(v) = \min_{i,l} \frac{c_{ni}^g d_{nl}}{z_{li}^g}, \quad d_{nl} \geq 1
\end{equation*}
where $z_{li}^g$ is home-production technology.
As $z_{li}^g \sim F(z)$, 
the share of expenditures by country $n$ on intermediate goods produced in country $l$ with country $i$ technology is:
\begin{equation}
    \pi_{n l i}^g = \frac{T_i\left(\tilde{c}_{n i}^g\right)^{-\theta}
            }{
                \sum_j T_j\left(\tilde{c}_{n j}^g\right)^{-\theta}} 
                \frac{\left(c_{l i}^g d_{n l}\right)^{-\theta /(1-\rho)}
                }{\sum_k\left(c_{k i}^g d_{n k}\right)^{-\theta /(1-\rho)}
                }
\end{equation}
where $\tilde{c}_{n i}^g=\left(\sum_k\left(c_{k i}^g d_{n k}\right)^{-\theta /(1-\rho)}\right)^{-(1-\rho) / \theta}$
This expression has a natural interpretation: 
The first term on the right-hand side is the share of expenditures that country $n$ allocates to intermediate goods produced with
country $i$’s technologies independently of the location where they are produced,
while the second term on the right-hand side is the share of these goods that are produced in country $l$. 

\subsubsection{Calibration}