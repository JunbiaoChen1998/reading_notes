\subsubsection{Motivation}
An insight from Ricardo (1817):
Two countries gain more from trade when they have dissimilar production productivities.
However, the models in the recent quatitative trade literature, building on EK (2002), relies on independence assumptions,
which, although leading to convenient functional forms for estimation,
restrict \textbf{expenditure substitution patterns} and 
\textbf{impact inference on the gains from trade}.

\subsubsection{Takeaway and Contribution}
\paragraph{Takeaway}
\begin{enumerate}
    \item This paper proposes a cross-nested CES structure for productivity, 
where the novelty comes from treating each nest as an unobserved—or, latent—dimension of the data.
% Particularly, this paper partitions sectors into multiple nests in a multisector Ricardian model,
% allowing rich cross-sector substition patterns in counterfactual analysis.
    \item This paper presents an alternative to the BLP procedure in \citep{Adao:2017}, but shares the same
    goal of departing from IIA.
\end{enumerate}

\paragraph{Contribution}
This paper generalizes the ACR's gains from trade to a GEV class.

\subsubsection{A GEV Framework}
\paragraph{Preferences}
Consumers have identical CES preferences over continuum of goods, $v \in [0,1]$ 
with elasticity of substitution $\eta > 1$.
Thus, the expenditure share on $v$ is 
\begin{equation}
    X_d(v) = \left(\frac{P_d(v)}{P_d}\right)^{1-\eta} X_d,
\end{equation}
where the subscript $d$ denotes destination countries.
$P_d(v) = \left(\int_0^1 P_d(v)^{1-\eta} dv \right)^{\frac{1}{1-\eta}}$
is the price level, and $X_d$ is total expenditure in country $d$.

\paragraph{Production}
Each good $v$ is produced with a labor-only technology that features constant returns to scale:
\begin{equation}
    Y_{od}(v) = Z_{od}(v)L_{od}(v),
\end{equation}
where $Z_{od}(v)$ is a generalization of the standard iceberg trade costs assumption 
that assumes $Z_{od} = \frac{Z_o(v)}{\tau_{od}}$.

\paragraph{Max-Stable Multivariate Frechet Productivity}
Given a destination country $d$,
this paper assumes that the joint distribution of productivity across 
origin countries is given by
\begin{equation}
    \text{Pr}\left[Z_{1 d}(v) \leq z_1, \ldots, Z_{N d}(v) \leq z_N\right]
    =
    \exp \left[-G^d\left(T_{1 d} z_1^{-\theta}, \ldots, T_{N d} z_N^{-\theta}\right)\right]    
\end{equation}

where $T_{o d}>0$ is the scale parameter and 
$\theta>0$ the shape parameter characterizing the marginal Fr\'{e}chet distributions, 
$\text{Pr}\left[Z_{o d}(v) \leq z\right]=e^{-T_{o d} z^{-\theta}}$. 
The scale parameters capture the absolute advantage of countries, 
while the shape parameter regulates the heterogeneity of iid productivity draws across the continuum of goods.%
\footnote{
    The smaller $\theta$ is, the greater the heterogeneity.
}

The function $G^d$ is a correlation function, also called tail dependence function in probability and statistics. This function allows for a flexible dependence