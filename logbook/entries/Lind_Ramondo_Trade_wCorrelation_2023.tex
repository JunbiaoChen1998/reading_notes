\subsection{Motivation}
An insight from Ricardo (1817):
Two countries gain more from trade when they have dissimilar production productivities.

However, the models in the recent quatitative trade literature relies on independence assumptions,
which, although leading to convenient functional forms for estimation,
restrict \textbf{expenditure substitution patterns} and 
\textbf{impact inference on the gains from trade}.

This paper proposes a cross-nested CES structure for productivity, 
where the novelty comes from treating each nest as an unobserved—or, latent—dimension of the data.
Particularly, this paper partitions sectors into multiple nests in a multisector Ricardian model,
allowing rich cross-sector substition patterns in counterfactual analysis.

