\subsubsection{Motivation}
An insight from Ricardo (1817):
Two countries gain more from trade when they have dissimilar production productivities.
However, the models in the recent quatitative trade literature, building on EK (2002), relies on independence assumptions,
which, although leading to convenient functional forms for estimation,
restrict \textbf{expenditure substitution patterns} and 
\textbf{impact inference on the gains from trade}.

\subsubsection{Takeaways and Contribution}
\paragraph{Takeaways}
\begin{enumerate}
    \item This paper proposes a cross-nested CES (CNCES) structure for productivity draws, 
    which treating the nests as latent factors in the multisector trade data.
    In the context of Ricardian theory, 
    these latent factors have a natural interpretation 
    that production and transportation technology may be shared across countries and sectors.
% Particularly, this paper partitions sectors into multiple nests in a multisector Ricardian model,
% allowing rich cross-sector substition patterns in counterfactual analysis.
    \item This paper presents an alternative to the BLP procedure in \citep{Adao:2017}, but shares the same
    goal of departing from IIA.
\end{enumerate}

\paragraph{Contribution}
This paper generalizes the ACR's gains from trade sufficient statistics to a GEV class.
% Generally, it is the correlation function that controls comparative advantage 
% since it determines the joint distribution of productivity between any two countries.
Assuming productivities are drawn from CNCES, the gains from trade (GFT) relative to autarky are
\begin{equation}
\frac{W_d / P_d}{W_d^A / P_d^A}=\pi_{d d}^{-1 / \theta}\left[\sum_{k=1}^K \frac{\left(\pi_{k d d}^W\right)^{1-\rho_k} \pi_{k d}^B}{\pi_{d d}}\right]^{-\frac{1}{\theta}}
\end{equation}
Notably, the GFT not only depends on own share $\pi_{dd}$, and shape parameter of Fr\'{e}chet $\theta$,
but also nest-own trade share $\pi_{kd}$ and correlation coefficients $\rho_s$.

\subsubsection{Theory}
\paragraph{Preferences}
Consumers have identical CES preferences over continuum of goods, $v \in [0,1]$ 
with elasticity of substitution $\eta > 1$.
Thus, the expenditure share on $v$ is 
\begin{equation}
    X_d(v) = \left(\frac{P_d(v)}{P_d}\right)^{1-\eta} X_d,
\end{equation}
where the subscript $d$ denotes destination countries.
$P_d(v) = \left(\int_0^1 P_d(v)^{1-\eta} dv \right)^{\frac{1}{1-\eta}}$
is the price level, and $X_d$ is total expenditure in country $d$.

\paragraph{Production}
Each good $v$ is produced with a labor-only technology that features constant returns to scale:
\begin{equation}
\label{eqn:production_technology}
    Y_{od}(v) = Z_{od}(v)L_{od}(v),
\end{equation}
where $Z_{od}(v)$ is a generalization of the standard iceberg trade costs assumption 
that assumes $Z_{od} = \frac{Z_o(v)}{\tau_{od}}$.

\paragraph{Max-Stable Multivariate Frechet Productivity}
Given a destination country $d$,
this paper assumes that the joint distribution of productivity across 
origin countries is given by
\begin{equation}
    \text{Pr}\left[Z_{1 d}(v) \leq z_1, \ldots, Z_{N d}(v) \leq z_N\right]
    =
    \exp \left[-G^d\left(T_{1 d} z_1^{-\theta}, \ldots, T_{N d} z_N^{-\theta}\right)\right]    
\end{equation}

where $T_{o d}>0$ is the scale parameter and 
$\theta>0$ the shape parameter characterizing the marginal Fr\'{e}chet distributions, 
$\text{Pr}\left[Z_{o d}(v) \leq z\right]=e^{-T_{o d} z^{-\theta}}$. 
The scale parameters capture the absolute advantage of countries, 
while the shape parameter regulates the heterogeneity of iid productivity draws across the continuum of goods.%
\footnote{
    The smaller $\theta$ is, the greater the heterogeneity.
}

The function $G^d$ is a correlation function, 
also called \textcolor{blue}{tail dependence function} 
in probability and statistics. 
This function allows for a flexible dependence structure across origin countries $o$ 
serving destination $d$.%
\footnote{
    See Gudendorf, Gordon, and Johan Segers. 2010. “Extreme-Value Copulas.”
}
In EK (2002), productivities are independent across origin countries, so 
$G^d(x_1, x_2, \ldots, x_N) = \sum_{o=1}^N x_o$,
and
\begin{equation}
    \begin{aligned}
        \text{Pr}\left[Z_{1 d}(v) \leq z_1, \ldots, Z_{N d}(v) \leq z_N\right] & 
        = \prod_{o=1,\ldots,N}  \text{Pr}\left[Z_{o d}(v) \leq z_o \right] \\
        & = \exp \left( - \sum_{o=1}^N T_{od} z_o^{-\theta} \right)
    \end{aligned}
\end{equation}

\textbf{Remark:} With independent productivities, 
$\theta$ alone governs the gains from trade in the EK model.



\begin{boxedproposition}[CNCES Approximation]
    Any correlation function can be approximated uniformly on compact sets using a cross-nested CES (CNCES) correlation function.
    \begin{equation}
        \label{eqn:cnces:approx}
        G^d\left(x_1, \ldots, x_N\right)=\sum_{s=1}^S\left[\sum_{o=1}^N\left(\omega_{s o d} x_o\right)^{\frac{1}{1-\rho_s}}\right]^{1-\rho_s},
    \end{equation}
where $s = 1,2,\ldots,S$ is the nest index. 
$\rho_s \in[0,1) \forall s$ is correlation coefficient that governs the correlation in productivity across origins within nest $s$. 
For $\rho_s=0$, productivity is independent and the $s^{th}$ nest is additive. 
In contrast, as $\rho_k \rightarrow 1$,
productivity becomes perfectly correlated within nest $s$, and the $s^{th}$ nest converges to a max function.
$\omega_{s o d}>0$, and $\sum_s \omega_{s o d}=1$. 
The weight $\omega_{s o d}$ indicates the relative importance of each nest $s$ for a given trading pair od. 
If $\omega_{s o d}$ is high, nest $s$ is particularly productive in country $o$ for delivery to $d$.
\\
\noindent \textbf{PROOF:}
\end{boxedproposition}

\textbf{Remarks on CNCES}
Suppose nests in equation~(\ref{eqn:cnces:approx})
corresponde to sectors $s = 1,2, \ldots, S$.
\begin{itemize}
    \item Productivity draws can be correlated $\rho_s \in[0,1)$ within sector.
    \item Correlation can be different across sectors: $\rho_s \neq \rho_{s'}$.
    \item Productivity draws are independent across sectors and, within sector, correlation is homogeneous across origins.
\end{itemize}

\subsubsection{Quantitative Application}
\cite{Lind:2023} generalize the production technology in equation~(\ref{eqn:production_technology})
to the following specification where 
$Z_{sod}(v)$ --- the productivity for a good $v$ in country $o$ sector $s$ sold to $d$ ---
is distributted according to 
\begin{equation}
    \text{Pr}\left[Z_{s o d}(v) \leq z_{s o}, \forall s, o\right] = 
    \exp \left[-\sum_{k=1}^K\left(\sum_{s=1}^S \sum_{o=1}^N\left(T_{k s o d}^* z_{s o}^{-\theta}\right)^{\frac{1}{1-\rho_k}}\right)^{1-\rho_k}\right]
\end{equation}