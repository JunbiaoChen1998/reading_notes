\subsubsection{Model}
The Caliendo-Parro model is a multi-industry extension of the EK (2002) Ricardian model of trade \citep{Antras:2022}.
\paragraph{Environment}
$N$ countries.
$J$ sectors.
\textbf{One factor of production, labor, which is mobile across sectors but not across countries}.%
\footnote{
    See~\cite{Caliendo:2018} for an extension to a spatial setting with mobile labor across locations.
}
CRS production function.
Output produced in all sectors can be used as intermediate input or a final good.

\paragraph{Preferences}
The representative consumer in each country has preferences over the output of the $S$ sectors given by:
\begin{equation}
    u(C_n) = \prod_{j=1}^J \left(C_n^j \right)^{\alpha_j}, \quad \text{where } \sum_{j} \alpha_j = 1
\end{equation}
where $C_n^j$ denotes non-tradable final consumption goods.

\begin{figure}[h!]
    \centering
    \caption{Roundabout Production}
    \begin{tikzpicture}[node distance=60mm, on grid]
        % Nodes arranged in a rectangular layout with light blue fill
        \node[draw, rectangle, fill=lightblue!50, minimum width=80mm, minimum height=15mm] (Q) 
            {$Q_n^j = 
            \underbrace{\left(\int y_n^j (w^j)^{\frac{\sigma_j-1}{\sigma_j}} d w^j \right)^{\frac{\sigma_j}{\sigma_j - 1}}}_{\text{Sourced from $i$ = 1,2,...,$N$. (``Regional Trade'')}} 
            = C_n^j + \sum_{k=1}^J \int M_n^{j,k}(w^k)dw^k$};
        \node[draw, rectangle, fill=lightblue!30, minimum width=20mm, minimum height=10mm, above right=of Q, yshift=5mm] (C) 
            {$u(C_n^j) = \prod_{j=1}^J \left(C_n^j \right)^{\alpha_j}$}; % Adjusted position
        \node[draw, rectangle, fill=lightblue!30, minimum width=100mm, minimum height=10mm, below left=of Q, yshift=-10mm] (y) 
            {$y_n^j(w^j) = z_n^j(w^j)\bigg[ l_n^j(w^j) \bigg]^{\gamma_n^j} \underbrace{\prod_{k=1}^J \bigg[ M_n^{k,j}(w^j) \bigg]^{\gamma_n^{k,j}}}_{\text{Sourced from $n$. (``Local Input-Output'')}}$};
    
        % Edges to represent the roundabout process with adjusted curvature
        \draw[->, very thick, >=Latex, bend left=30] (Q) to 
            node[midway, left=1mm, align=center, text width=70mm] 
            {price of final goods $C_n^j$: \\ $P_n^j = A^j \bigg[\sum_{i}\lambda_i^j(c_i^j \kappa_{ni}^j)^{-\theta^j} \bigg]^{-\frac{1}{\theta^j}}$}
            (C);
        \draw[->, very thick, >=Latex, bend left=40] (Q) to 
            node[midway, right=1mm, align=center, text width=70mm] 
            {price of composite goods $M_n^j$: \\ 
            $P_n^j = A^j \bigg[\sum_{i}\lambda_i^j(c_i^j \kappa_{ni}^j)^{-\theta^j} \bigg]^{-\frac{1}{\theta^j}}$} 
            (y);
        \draw[->, very thick, >=Latex, bend left=30] (y) to 
            node[midway, left=1mm, align=center, text width=70mm] 
            {price of $w^j$ in $n$: \\ $ p_n^j(w^j) = \min_i \frac{\kappa_{ni}^j c_{ni}^j}{z_i^j(w^j)}$, \\ 
            $c_{ni}^j = (w_n)^{\gamma_n^j} \prod_k (P_n^k)^{\gamma_{n}^{kj}}$} 
            (Q);
    \end{tikzpicture}
\end{figure}

\paragraph{Intermediate goods}%
\footnote{
    \cite{Antras:2022} term the production function in \cite{Caliendo:2015} as a ``roundabout model''
    and suggest that ``it has quickly become a benchmark model in the field [of modeling GVCs with macro approaches].
    The macro approaches emphasize the role of trade in intermediate inputs and of global inter-sectoral linkages 
    in shaping response of the world economy to various types of shocks.''
}
A continuum of intermediate goods $\omega^j \in [0, 1]$ is produced in each sector $j$.
Two types of inputs, labor and ``materials from all sectors'' are used for the 
production of each $\omega^j$.
The production technology of a good $\omega^j$ is 
\begin{equation}
    y_n^j(w^j) = z_n^j(w^j)\bigg[ l_n^j(w^j) \bigg]^{\gamma_n^j} \prod_{k=1}^J \bigg[ M_n^{k,j}(w^j) \bigg]^{\gamma_n^{k,j}}
\end{equation}
where
\begin{enumerate}
    \item ``$z_n^j(w^j) \sim \text{Frechet}$'' governs the efficiency of producing intermediate good $\omega^j$ in county $n$
    \item $l_n^j(w^j)$ is labor
    \item $M_n^{k,j}(w^j)$ is the materials from sector $k$ used for the production of intermediate good $\omega^j$ (capturing the roundabout production)
    \item $\gamma_n^{k,j} \geq 0$ is the share of the materials used from $k$ in $j$ (capturing the strength of the I-O linkages)
\end{enumerate}

\paragraph{Composite goods: CES across goods}%
Producers of composite goods in sector $j$ and country $n$ supply final consumptions $C_n^j$ and materials $M_n^{k,j}$
using a CES technology across goods:
\begin{equation}
    Q_n^j = \left(\int q_n^j (w^j)^{\frac{\sigma_j-1}{\sigma_j}} d w^j \right)^{\frac{\sigma_j}{\sigma_j - 1}}
\end{equation}
where $\sigma^j > 1$ is the elasticity of substitution across intermediate goods within sector $j$,
and $q_n^j (w^j)$ is the demand of intermediate goods $w^j$ from the lowest cost supplier.

\paragraph{Trade costs}
There are two types of trade costs: iceberg trade costs and an ad-valorem flat-rate tariffs:
\begin{equation}
    \kappa_{ni}^j = (1 + \tau_{ni}^j) d_{ni}^j = \tilde{\tau}_{ni}^j d_{ni}^j 
\end{equation}


\paragraph{Prices and Expenditure}
Since production of intermediate goods is at CRS and markets are perfectly competitive,
firms price at unit cost, $c_n^j / z_n^j(w^j)$,
where $c_n^j$ denotes the costs of an input bundle.
In particular,
\begin{equation}
    c_n^j = \Gamma_n^j \times (w_n^j)^{\gamma_n^j} \prod_{k=1}^J (P_{n}^k)^{\gamma_n^{k,j}}
\end{equation}

Given trade cost structure, the price of $\omega^j$ in $n$ is 
\begin{equation}
    p_n^j(w^j) = \min_i \frac{\kappa_{ni}^j c_n^j}{z_i^j(w^j)}
\end{equation}


\subsubsection{Calibration}
\paragraph{Dispersion of productivities (or trade elasticities)}