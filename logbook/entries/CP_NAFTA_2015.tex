\subsubsection{Model}
The Caliendo-Parro model is a multi-industry extension of the EK (2002) Ricardian model of trade \citep{Antras:2022}.
\paragraph{Environment}
$N$ countries.
$J$ sectors.
One factor of production, labor.

\paragraph{Households}
\begin{equation}
    u(C_n) = \prod_{j=1}^J \left(C_n^j \right)^{\alpha_j}, \quad \text{where } \sum_{j} \alpha_j = 1
\end{equation}
where $C_n^j$ denotes non-tradable a final (consumer) good.%
\footnote{
    \cite{Antras:2022} refer $C_n^j$ as a consumer good.
}

\paragraph{Intermediate goods}%
\footnote{
    \cite{Antras:2022} term the production function in \cite{Caliendo:2015} as a ``roundabout model''
    and suggest that ``it has quickly become a benchmark model in the field [of modeling GVCs with macro approaches].
    The macro approaches emphasize the role of trade in intermediate inputs and of global inter-sectoral linkages 
    in shaping response of the world economy to various types of shocks.''
}
A continuum of intermediate goods $\omega^j \in [0, 1]$ is produced in each sector $j$.
Two types of inputs, labor and ``materials'' from all sectors are used for the 
production of each $\omega^j$.
The production technology of a good $\omega^j$ is 
\begin{equation}
    q_n^j(w^j) = z_n^j(w^j)\bigg[ l_n^j(w^j) \bigg]^{\gamma_n^j} \prod_{k=1}^J \bigg[ m_n^{k,j}(w^j) \bigg]^{\gamma_n^{k,j}}
\end{equation}
where
\begin{enumerate}
    \item $z_n^j(w^j)$ governs the efficiency of producing intermediate good $\omega^j$ in county $n$
    \item $l_n^j(w^j)$ is labor
    \item $m_n^{k,j}(w^j)$ is the materials from sector $k$ used for the production of intermediate good $\omega^j$ (capturing the roundabout production)
    \item $\gamma_n^{k,j} \geq 0$ is the share of the materials used from $k$ in $j$ (capturing the strength of the I-O linkages)
\end{enumerate}