In this section, I summarize the multinational production literature that 
explicitly model the role of platform by multinational firms.%
% \footnote{
% This section is largely based on Felix Tintelnot's lecture notes.
% }
These flows are important because 
\begin{enumerate}
    \item A large share of foreign affiliates' production is exported to other countries, beyond exports back to their parent countries \citep{Tintelnot:2017}.
    \item They also lead to interesting implications for commercial policy and for the welfare gains from trade and multinational production.
\end{enumerate}

The main challenge for modeling export platforms is the ugly corner solutions for production, consumption, and trade.
However, it is not surprising that,
the same solution that EK (2002) provided for extending the Ricardian trade model to multiple countries is 
also helpful for the extension of the multinational production models to multiple countries:
\textbf{A probabilistic structure based on the Frechet distribution.}
\begin{equation}
    \text{Pr}( Z_i(j) \leq z) = F_i(z) = \exp(-T_i z^{-\theta})
\end{equation}
where 
\begin{itemize}
    \item $T_i > 0$ governs the \textbf{location} of $F_i(z)$. 
Higher $T_i$ implies higher productivity draw is more likely for any good $j$. ``Absolute advantage.''
    \item $\theta > 1$ determines the dispersion of productivity, where a higher $\theta$ means there is less dispersion (common across countries).
``Comparative advantage.''
\end{itemize}